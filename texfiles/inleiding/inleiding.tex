\chapter{Financieel plan}
\label{financieelplan}

Uitleg: Omschrijf kort de functie die je zal uitoefenen. Bepaal het \textbf{loon} dat je daar zal krijgen als afgestudeerde bachelor student informatica. Wordt je loon
jaarlijks geïndexeerd? Verwacht je opslag na een aantal jaren?

In de lessen bespreken we hoe je een persoonlijk financieel plan schrijft. We starten met het
schrijven van je persoonlijke financieel plan (oefening 4.3). Je werkt dit verder uit, en steekt je
persoonlijk financieel plan in deel 1 van je portfolio.\newline\newline

Ik begin te werken in juli 2016 als \functiejob in een bedrijf.


Als je niet goed weet wat een masterproef is, kan je altijd Wikipedia\cite{Wikipedia} eens nakijken.

%\section{Lorem ipsum 4--5}
%\lipsum[1-2]

\newpage
\section{Budget\index{budget}}

Stel voor het eerste jaar een \textbf{budget} op. Gebruik het voorbeeld dat je kan vinden in het handboek beleggingsleer pagina 120. Bepaal zelf of je een vriend(in) hebt of dat je alleenstaand bent. Een aantal opmerkingen:
\begin{itemize}
	\item heb je een vriend(in), bespreek dan kort haar persoonlijke situatie (leeftijd,werk, loon,…).
	\item ga je nog thuis blijven wonen het eerste jaar of ga je een appartement of huis huren? Zoek dan zelf op websites een geschikte woning, en neem de
	huurprijs op in je budget.
	\item raam je inkomsten en uitgaven en vermeldt duidelijk je bronnen en de
	veronderstellingen die je inbouwt.
\end{itemize}

\subsection{Inkomsten}\index{Inkomsten}


\begin{comment}

\newpage
\renewcommand{\arraystretch}{1.5}
%The height of each row is set to 1.5 relative to its default height.

\begin{table}[!h]
	\centering
	%\begin{tabular}{|l|l|l|l|}
	\begin{tabular}{l l}
		\arrayrulecolor{black}
		\hline
		%\toprule
		\rowcolor{purpleexcel}
		\multicolumn{2}{c}{\textcolor{white}{\textbf{BUDGET}}} \\ \hline
		
		\rowcolor{purpleexcel}
		\multicolumn{1}{c}{\textcolor{white}{\textbf{Inkomsten}}} & \multicolumn{1}{c}{\textcolor{white}{\textbf{euro}}} \\ \hline
		%\centering Inkomsten & \centering euro & \centering Uitgaven & \centering euro \tabularnewline \hline
		
		\textbf{1. Beroepsinkomen}                      & \textbf{1535 \euro{}}  \\ \hline
		Beroepsinkomen man                              & 1535 \euro{}          \\ \hline
		Beroepsinkomen vrouw                            & 0  \euro{}            \\ \hline
		Maaltijdcheques                                 & 0 \euro{}             \\ \hline
		Dertiende maand en vakantiegeld                 & 0 \euro{}             \\ \hline
		Bonussen                                        & 0 \euro{}             \\ \hline
		Inkomsten uit zelfstandige activiteit           & 0 \euro{}             \\ \hline
		...                                             &                       \\ \hline
														&                       \\ \hline
		
		\textbf{2. Andere inkomsten}                    & 0 \euro{}             \\ \hline
		kindergeld                                      & 0 \euro{}             \\ \hline
		Huurinkomsten                                   & 0 \euro{}             \\ \hline
		Teruggave van belastingen                       & 0 \euro{}             \\ \hline
		Intresten, divendenden van beleggingen, ...     & 0 \euro{}             \\ \hline
														&                       \\ \hline
		
		&                                                                      \\ \hline
		\textbf{Totaal van de inkomsten}              & \textbf{1535 \euro{}}   \\ %\hline
		%                                        &               &                                               &       \\
		%\multicolumn{4}{|c|}{} \\ %\hline
		
		
	\end{tabular}
	\caption{budget}
	\label{tab:budgetoverzicht}
\end{table}

\end{comment}

\begin{table}[!htbp]
	\centering
	\begin{tabular}{@{}lr@{}} \toprule
		\multicolumn{2}{c}{Beroepsinkomen} \\ \cmidrule(r){1-2}
		Inkomen    										& Euro (\euro{})\\ \midrule
		Beroepsinkomen man      						& 13.65 \\
		Beroepsinkomen vrouw    						& 0.01 \\
		Maaltijdcheques       							& 92.50 \\
		Dertiende maand en vakantiegeld  				& 33.33 \\
		Bonussen 										& 8.99 \\
		\textbf{Totaal van de inkomsten}              	& \textbf{148,47 \euro{}}   \\ \bottomrule
	\end{tabular}
	\caption{Een tabel zoals het beter is.}
	\label{tab:juist}
\end{table}

\begin{table}[!htbp]
	\centering
	\begin{tabular}{@{}lr@{}} \toprule
		\multicolumn{2}{c}{Andere inkomsten} \\ \cmidrule(r){1-2}
		Inkomen    										& Euro (\euro{})\\ \midrule
		kindergeld      								& 0 \\
		Huurinkomsten    								& 0 \\
		Teruggave van belastingen       				& 0 \\
		Intresten, divendenden van beleggingen, ...  	& 0 \\
		\textbf{Totaal van de inkomsten}              	& \textbf{0 \euro{}}   \\ \bottomrule
	\end{tabular}
	\caption{Een tabel zoals het beter is.}
	\label{tab:juist}
\end{table}


% Reset the margins to be symmetric
%\setlrmarginsandblock{1.5cm}{1.5cm}{*}
%\setulmarginsandblock{2cm}{2cm}{*}
%\checkandfixthelayout


%\tabelgeometry

\begingroup

%\setlength{\arrayrulewidth}{0.6mm}
%This sets the thickness of the borders of the table. In the example is 1mm but you can use other units, see the article Lengths in LaTeX for a complete list.
%\setlength{\tabcolsep}{18pt}
%The space between the text and the left/right border of its containing cell is set to 18pt with this command. Again, you may use other units if needed.
\renewcommand{\arraystretch}{1.5}
%The height of each row is set to 1.5 relative to its default height.

\begin{table}[!htbp]
	\centering
	%\begin{tabular}{|l|l|l|l|}
	\begin{tabular}{l l l l}
		\arrayrulecolor{black}
		\hline
		%\toprule
		\rowcolor{purpleexcel}
		\multicolumn{4}{c}{\textcolor{white}{\textbf{BUDGET}}} \\ \hline
		
		\rowcolor{purpleexcel}
		\multicolumn{1}{c}{\textcolor{white}{\textbf{Inkomsten}}} & \multicolumn{1}{c}{\textcolor{white}{\textbf{euro}}} & 
		\multicolumn{1}{c}{\textcolor{white}{\textbf{Uitgaven}}} & \multicolumn{1}{c}{\textcolor{white}{\textbf{euro}}} \\ \hline
		%\centering Inkomsten & \centering euro & \centering Uitgaven & \centering euro \tabularnewline \hline
		
		\textbf{1. Beroepsinkomen}                      & \textbf{1535 \euro{}} & \textbf{1. Maandelijks terugkerende uitgaven} & 1005 \euro{} \\ \hline
		Beroepsinkomen man                              & 1535 \euro{}          & Huishuur of afbetaling van de lening          & 0 \euro{} \\ \hline
		Beroepsinkomen vrouw                            & 0  \euro{}            & Uitgven voor voeding en kleding               & 700 \euro{} \\ \hline
		Maaltijdcheques                                 & 0 \euro{}             & Elektriciteit, water, gas, internet, ...      & 305 \euro{} \\ \hline
		Dertiende maand en vakantiegeld                 & 0 \euro{}             & ...                                           & 0 \euro{} \\ \hline
		Bonussen                                        & 0 \euro{}             &                                               &  \\ \hline
		Inkomsten uit zelfstandige activiteit           & 0 \euro{}             &                                               &  \\ \hline
		...                                             &                       &                                               &   \\ \hline
		&                       &                                               & \\ \hline
		
		\textbf{2. Andere inkomsten}                    & 0 \euro{}             & \textbf{2. Jaarlijks terugkerende uitgaven}   & 970 \euro{} \\ \hline
		kindergeld                                      & 0 \euro{}             & Brand- en autoverzekering                     & 720 \euro{} \\ \hline
		Huurinkomsten                                   & 0 \euro{}             & Onderhoud auto                                & 250 \euro{} \\ \hline
		Teruggave van belastingen                       & 0 \euro{}             & Onderhoud woning                              & 0 \euro{} \\ \hline
		Intresten, divendenden van beleggingen, ...     & 0 \euro{}             & Studies kinderen                              & 0 \euro{} \\ \hline
		&                       & Belastingen                                   & 0 \euro{} \\ \hline
		&                       & Vakantie                                      & 0 \euro{} \\ \hline
		&                       & Diversen                                      & 0 \euro{} \\ \hline
		&                       & ...                                           & 0 \euro{} \\ \hline
		&                       &                                               &   \\ \hline
		&                       &                                               &   \\ \hline
		&                       &                                               &   \\ \hline
		\textbf{Totaal van de inkomsten}              & \textbf{1535 \euro{}} & \textbf{Totaal van de uitgaven \euro{}}       & \textbf{1975} \\ %\hline
		%                                        &               &                                               &       \\
		%\multicolumn{4}{|c|}{} \\ %\hline
		%\multicolumn{4}{|c|}{} \\ %\hline
		%Totaal van de inkomsten & 1535
		\bottomrule
		\rowcolor{red}
		\multicolumn{4}{c}{\textcolor{white}{\textbf{Spaarvermogen = -440 \euro{}}}} \\ \hline
		
	\end{tabular}
	\caption{budget}
	\label{tab:budgetoverzicht}
\end{table}
%\restoregeometry

\endgroup



%--------------------------------------------------------
%--------------------------------------------------------
%--------------------------------------------------------

\section{Vermogensbalans}



%\newgeometry{total={7.7in, 9.5in},top=0.7in,bottom=0.7in}

\begingroup


%\setlength{\arrayrulewidth}{0.6mm}
%This sets the thickness of the borders of the table. In the example is 1mm but you can use other units, see the article Lengths in LaTeX for a complete list.
%\setlength{\tabcolsep}{18pt}
%The space between the text and the left/right border of its containing cell is set to 18pt with this command. Again, you may use other units if needed.
\renewcommand{\arraystretch}{1.5}
%The height of each row is set to 1.5 relative to its default height.

\begin{table}[!htbp]
	\centering
	%\begin{tabular}{|l|l|l|l|}
	\begin{tabular}{l l l l}
%		\begin{tabular}{p{6cm} l p{5cm} l}
		\arrayrulecolor{black}
		\hline
		%\toprule
		\rowcolor{purpleexcel}
		\multicolumn{4}{c}{\textcolor{white}{\textbf{BUDGET}}} \\ \hline
		
		\rowcolor{purpleexcel}
		\multicolumn{1}{c}{\textcolor{white}{\textbf{Inkomsten}}} & \multicolumn{1}{c}{\textcolor{white}{\textbf{euro}}} & 
		\multicolumn{1}{c}{\textcolor{white}{\textbf{Uitgaven}}} & \multicolumn{1}{c}{\textcolor{white}{\textbf{euro}}} \\ \hline
		%\centering Inkomsten & \centering euro & \centering Uitgaven & \centering euro \tabularnewline \hline
		
		\textbf{1. Beroepsinkomen}                      & \textbf{1535 \euro{}} & \textbf{1. Maandelijks terugkerende uitgaven} & 1005 \euro{} \\ \hline
		Beroepsinkomen man                              & 1535 \euro{}          & Huishuur of afbetaling van de lening          & 0 \euro{} \\ \hline
		Beroepsinkomen vrouw                            & 0  \euro{}            & Uitgven voor voeding en kleding               & 700 \euro{} \\ \hline
		Maaltijdcheques                                 & 0 \euro{}             & Elektriciteit, water, gas, internet, ...      & 305 \euro{} \\ \hline
		Dertiende maand en vakantiegeld                 & 0 \euro{}             & ...                                           & 0 \euro{} \\ \hline
		Bonussen                                        & 0 \euro{}             &                                               &  \\ \hline
		Inkomsten uit zelfstandige activiteit           & 0 \euro{}             &                                               &  \\ \hline
		...                                             &                       &                                               &   \\ \hline
		&                       &                                               & \\ \hline
		
		\textbf{2. Andere inkomsten}                    & 0 \euro{}             & \textbf{2. Jaarlijks terugkerende uitgaven}   & 970 \euro{} \\ \hline
		kindergeld                                      & 0 \euro{}             & Brand- en autoverzekering                     & 720 \euro{} \\ \hline
		Huurinkomsten                                   & 0 \euro{}             & Onderhoud auto                                & 250 \euro{} \\ \hline
		Teruggave van belastingen                       & 0 \euro{}             & Onderhoud woning                              & 0 \euro{} \\ \hline
		Intresten, divendenden van beleggingen, ...     & 0 \euro{}             & Studies kinderen                              & 0 \euro{} \\ \hline
		&                       & Belastingen                                   & 0 \euro{} \\ \hline
		&                       & Vakantie                                      & 0 \euro{} \\ \hline
		&                       & Diversen                                      & 0 \euro{} \\ \hline
		&                       & ...                                           & 0 \euro{} \\ \hline
		&                       &                                               &   \\ \hline
		&                       &                                               &   \\ \hline
		&                       &                                               &   \\ \hline
		\textbf{Totaal van de inkomsten}              & \textbf{1535 \euro{}} & \textbf{Totaal van de uitgaven \euro{}}       & \textbf{1975} \\ %\hline
		%                                        &               &                                               &       \\
		%\multicolumn{4}{|c|}{} \\ %\hline
		%\multicolumn{4}{|c|}{} \\ %\hline
		%Totaal van de inkomsten & 1535
		\bottomrule
		\rowcolor{red}
		\multicolumn{4}{c}{\textcolor{white}{\textbf{Spaarvermogen = -440 \euro{}}}} \\ \hline
		
	\end{tabular}
	\caption{Vermogensbalans}
	\label{tab:vermogensbalans}
\end{table}

\endgroup

%\restoregeometry


%--------------------------------------------------------
%--------------------------------------------------------
%--------------------------------------------------------



\section{Spaarvermogen}

Uitleg: Bereken je \textbf{spaarvermogen} per jaar. Bespreek kort hoe je dit zal beleggen, bereken je jaarlijkse return en voeg dit bij aan de inkomsten van het jaar erop.\newline\newline

Ik ga 10000 euro sparen over een periode van 10 jaar. Ik heb dit berekend met \fnurl{de spaarsimulator op spaargids.be}{http://www.spaargids.be/sparen/spaarsimulator.html} en ook via \fnurl{de kbc spaarsimulator}{https://www.kbc.be/PBL/CC028/spaarsimulator} en via \fnurl{de ING spaarsimulator}{https://www.ing.be/nl/retail/savings-calculator}.

\begin{figure}[!htbp]
	\centering
	\includegraphics[width=6in]{kbcspaarsimulator.PNG}
	\caption{Kbc spaarsimulator}
	\label{fig:Kbc spaarsimulator}
\end{figure}

%--------------------------------------------------------
%--------------------------------------------------------
%--------------------------------------------------------

% Reset the margins to be symmetric
\setlrmarginsandblock{3.5cm}{3.5cm}{*}
%\setulmarginsandblock{2cm}{2cm}{*}
\checkandfixthelayout

\section{Vermogensbalans na 5 jaar}

Stel je \textbf{vermogensbalans} op na 5 jaar, dus op datum van 01 juli 2022. Gebruik het voorbeeld dat je kan vinden in het handboek beleggingsleer pagina 119.

\begingroup

%\setlength{\arrayrulewidth}{0.6mm}
%This sets the thickness of the borders of the table. In the example is 1mm but you can use other units, see the article Lengths in LaTeX for a complete list.
%\setlength{\tabcolsep}{18pt}
%The space between the text and the left/right border of its containing cell is set to 18pt with this command. Again, you may use other units if needed.
\renewcommand{\arraystretch}{1.5}
%The height of each row is set to 1.5 relative to its default height.

\begin{table}[!htbp]
	\centering
	%\begin{tabular}{|l|l|l|l|}
	\begin{tabular}{l l l l}
		\arrayrulecolor{black}
		\hline
		%\toprule
		\rowcolor{purpleexcel}
		\multicolumn{4}{c}{\textcolor{white}{\textbf{BUDGET}}} \\ \hline
		
		\rowcolor{purpleexcel}
		\multicolumn{1}{c}{\textcolor{white}{\textbf{Inkomsten}}} & \multicolumn{1}{c}{\textcolor{white}{\textbf{euro}}} & 
		\multicolumn{1}{c}{\textcolor{white}{\textbf{Uitgaven}}} & \multicolumn{1}{c}{\textcolor{white}{\textbf{euro}}} \\ \hline
		%\centering Inkomsten & \centering euro & \centering Uitgaven & \centering euro \tabularnewline \hline
		
		\textbf{1. Beroepsinkomen}                      & \textbf{1535 \euro{}} & \textbf{1. Maandelijks terugkerende uitgaven} & 1005 \euro{} \\ \hline
		Beroepsinkomen man                              & 1535 \euro{}          & Huishuur of afbetaling van de lening          & 0 \euro{} \\ \hline
		Beroepsinkomen vrouw                            & 0  \euro{}            & Uitgven voor voeding en kleding               & 700 \euro{} \\ \hline
		Maaltijdcheques                                 & 0 \euro{}             & Elektriciteit, water, gas, internet, ...      & 305 \euro{} \\ \hline
		Dertiende maand en vakantiegeld                 & 0 \euro{}             & ...                                           & 0 \euro{} \\ \hline
		Bonussen                                        & 0 \euro{}             &                                               &  \\ \hline
		Inkomsten uit zelfstandige activiteit           & 0 \euro{}             &                                               &  \\ \hline
		...                                             &                       &                                               &   \\ \hline
		&                       &                                               & \\ \hline
		
		\textbf{2. Andere inkomsten}                    & 0 \euro{}             & \textbf{2. Jaarlijks terugkerende uitgaven}   & 970 \euro{} \\ \hline
		kindergeld                                      & 0 \euro{}             & Brand- en autoverzekering                     & 720 \euro{} \\ \hline
		Huurinkomsten                                   & 0 \euro{}             & Onderhoud auto                                & 250 \euro{} \\ \hline
		Teruggave van belastingen                       & 0 \euro{}             & Onderhoud woning                              & 0 \euro{} \\ \hline
		Intresten, divendenden van beleggingen, ...     & 0 \euro{}             & Studies kinderen                              & 0 \euro{} \\ \hline
		&                       & Belastingen                                   & 0 \euro{} \\ \hline
		&                       & Vakantie                                      & 0 \euro{} \\ \hline
		&                       & Diversen                                      & 0 \euro{} \\ \hline
		&                       & ...                                           & 0 \euro{} \\ \hline
		&                       &                                               &   \\ \hline
		&                       &                                               &   \\ \hline
		&                       &                                               &   \\ \hline
		\textbf{Totaal van de inkomsten}              & \textbf{1535 \euro{}} & \textbf{Totaal van de uitgaven \euro{}}       & \textbf{1975} \\ %\hline
		%                                        &               &                                               &       \\
		%\multicolumn{4}{|c|}{} \\ %\hline
		%\multicolumn{4}{|c|}{} \\ %\hline
		%Totaal van de inkomsten & 1535
		\bottomrule
		\rowcolor{red}
		\multicolumn{4}{c}{\textcolor{white}{\textbf{Spaarvermogen = -440 \euro{}}}} \\ \hline
		
	\end{tabular}
	\caption{Vermogensbalans na 5 jaar}
	\label{tab:vermogensbalans_na_5_jaar}
\end{table}

\endgroup


%--------------------------------------------------------
%--------------------------------------------------------
%--------------------------------------------------------

\section{Doelstellingen}

Wat zijn de \textbf{doelstellingen} op korte, middellange en lange termijn? Wil je sparen
voor een auto of een grote reis? Wil je binnen een aantal jaren een woning of
appartement kopen? Als je verschillende doelstellingen hebt, welke zijn dan het
belangrijkste? Welke bedragen heb je hiervoor nodig? Maak een actieplan op om
deze doelstelling te halen.

\subsection{Doelstellingen op korte termijn}

\lipsum[1-5]

\subsection{Doelstellingen op middellange termijn}

\lipsum[1-5]

\subsection{Doelstellingen op lange termijn}

\lipsum[1-5]

%--------------------------------------------------------
%--------------------------------------------------------
%--------------------------------------------------------

%--------------------------------------------------------
%--------------------------------------------------------
%--------------------------------------------------------

%%% Local Variables: 
%%% mode: latex
%%% TeX-master: "masterproef"
%%% End: 
