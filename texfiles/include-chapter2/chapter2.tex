\chapter{Beleggingsportefeuille}
\label{hoofdstuk:beleggingsportefeuille}
\thispagestyle{chapternohead}
\index{Beleggingsportefeuille}

%Uitleg: Je hebt een beleggingsportefeuille van 50.000 EUR. Beleg deze 50.000 EUR, licht je keuzes toe en steek de productfiches in bijlage. In de lessen bespreken we de verschillende beleggingsproducten in detail. We gaan op zoek naar producten die op dit moment beschikbaar zijn op de markt, bespreken sites die je helpen met de selectie,… Zorg voor een visuele voorstelling van je portefeuille en wees hierbij creatief!

%$\Rightarrow$ productfiches komen in bijlage

%\todo[inline]{Uitleg hierboven op het einde nog verwijderen.}

\begin{comment}
\begin{figure}[!htbp]
	\centering
	\includegraphics[width=10cm]{images/producten/overzichtbelegdbedrag}
	\caption{Overzicht verdeling van het verdeelde bedrag per product in percentage}	
\label{fig:overzichtbelegdbedrag}
	\end{figure}
\end{comment}

\section{Product 1: de spaarrekening (financiële buffer)}

Ik heb de spaarrekening gekozen als eerste product omdat een spaarrekening de meest liquide beleggingsvorm is en je altijd direct aan je geld aankan. De spaarrekening is ook gratis en veilig omdat ze eeuwig kan blijven bestaan. Ze is ook fiscaal voordelig want voor natuurlijke personen is de eerste schijf van 1880~\euro{} rente per jaar vrijgesteld van roerende voorheffing, die momenteel 15\% bedraagt.

Ik heb gekozen voor de \fnurl{KBC spaarrekening}{https://www.kbc.be/product/sparen/spaarrekeningen/spaarrekening.html} omdat ik zelf al mijn hele leven bij KBC aangesloten ben en graag mijn geld zou willen beleggen in een bank dat ik vertrouw. Deze spaarrekening heeft een veiligheidsrating van "\textbf{A1} op de Moody's schaal en \textbf{A-} op de Fitch's schaal\footnote{http://www.spaargids.be/sparen/spaartarieven.html} ".
%\todo[inline]{Is dit juist van de kosten?}

Hierop heb ik 13135,12~\euro{}, (= 50000 - de som van de bedragen van product 2,3 en 4), belegd aan een intrestpercentage van 0,11\%. Door dit lage percentage is de return ook niet zo hoog.


%\index{Return!Product 1}
%\index{Fiscaliteit!Product 1}
%\index{Kosten!Product 1}

%\todo[inline]{Wat is het intrestbedrag en percentage?}
%De return is niet zo hoog omdat spaarrekeningen niet zoveel meer opbrengen deze tijden.
%\newline\newline
%Altijd onderbouwen bij elk product:
%\begin{enumerate}
%	\item wat is de \textbf{veiligheid} van dit product?
%	\item wat is de \textbf{fiscaliteit} van dit product?
%\end{enumerate}

\subsection{Conclusie}
%Wat zijn de voor- en nadelen van dit product?

%De voordelen en nadelen komen uit het handboek pagina 13:

\begin{table}[!htbp]
	\centering
%	\begin{tabular}{@{}lr@{}}
		\begin{tabular}{c|c}
			\toprule
			
			 \cellcolor{lightgray}            +              &  \cellcolor{lightgray}                                      -                                        \\ \midrule
			 Toegankelijk en flexibel   &          Rentevergoeding onzeker (kan op elk moment worden  gewijzigd)          \\ \hline
			  Geld altijd beschikbaar   & Eerder als wachtrekening (op lange termijn producten met een hogere vergoeding) \\ \hline
			Fiscaal interessant systeem &  \\ \hline
			        Geen kosten         &  \\ \bottomrule
		\end{tabular}
	\caption{Voor-~en nadelen van de spaarrekening.}
	\label{tab:voordelenennadelenspaarrekening}
\end{table}


%-------------------------------------------------------------

\section{Product 2: Aandeel van Proximus }

%\todo[inline]{productfiche nog opzoeken en bijvoegen als bijlage en een referentie ernaar toe plaatsen in deze section}


Ik heb mij geïnformeerd voor de aandelen via de %BEL20\footnote{http://www.tijd.be/beurzen/euronext-brussel/bel20}
\fnurl{BEL20}{http://www.tijd.be/beurzen/euronext-brussel/bel20} %\url{http://www.tijd.be/beurzen/euronext-brussel/bel20}
. Op deze website heb ik besloten om aandelen van Proximus te kopen. Dit is een bedrijf dat genoteerd is op de BEL20 beurs. Proximus heeft als activiteiten internet, Telecommunicatie en televisie toegang te voorzien. Ze zitten in de Fixed Line Telecommunications sector dus. De aanbeveling van de analisten op \url{http://www.tijd.be/beurzen/Proximus.60187045#analyse} had een waardering van $3/5$ en was voor mij voldoende om overtuigd te zijn. Ik ga dus voor 12488~\euro{} aandelen kopen. Er is ook geen sprake van bijkomende kosten en ze delen elk jaar een dividend uit per aandeel.

Aandelen zijn wel niet altijd veilig doordat er koersschommelingen kunnen optreden op de beurs waardoor de waarde van je aandeel kan zakken of stijgen. Als er een verlies optreedt bij de aandelen, wordt er normaal ook geen dividend uitgekeerd. Aandelen zijn dus niet volledig veilig.

De liquiditeit van aandelen hangt af van de markt waarop ze genoteerd zijn en de beurskapitalisatie ervan.

De return van de aandelen bestaat uit de dividenden, aangevuld met een meer- of minder-warde bij verkoop van de aandelen. Het divident zelf is verschillend van jaar tot jaar, en is dus afhankelijk van de gerealiseerde winst.

Dividenden worden in België belast aan 25\%.

Aandelen hebben ook extra kosten omdat de belegger bij het uitvoeren van een beursorder, zowel bij aan- als verkoop, transactiekosten betalen die bestaan uit een beurstaks en makelaarsloon.

%\todo[inline]{Nog veiligheid, liquiditeit, return, fiscaliteit en kosten in tekst verwerken}

\begin{comment}

Altijd onderbouwen bij elk product:
\begin{enumerate}
	\item \textbf{waarom} heb je dit product gekozen?
	\item wat is de \textbf{veiligheid} van dit product?
	\item wat is de \textbf{liquiditeit} van dit product?
	\item wat is de \textbf{return} van dit product?
	\item wat is de \textbf{fiscaliteit} van dit product?
	\item wat zijn de \textbf{kosten} van dit product?
\end{enumerate}
\end{comment}
\
\begin{figure}[tbph!]
	\centering
	%\includegraphics[width=18cm]{images/aandelenproximusgekocht19december2016}
	\caption[Aandelen Proximus]{Aandelen Proximus gekocht op 19 december 2016}
	\label{fig:aandelenproximusgekocht19december2016}
\end{figure}


\subsection{Conclusie}
%Wat zijn de voor-~ en nadelen van dit product?

\begin{table}[!htbp]
	\centering
%	\begin{tabular}{@{}lr@{}}
		\begin{tabular}{p{10cm}|c}
			\toprule
			
			 \cellcolor{lightgray}            +              &  \cellcolor{lightgray}                                      -                                        \\ \midrule
			 Historisch gezien leveren aandelen op lange termijn de beste return, vergeleken met andere beleggingsvormen   & Return onzeker          \\ \hline
			  Geen meerwaardebelasting  & Mogelijkheid op minderwaarden \\ \hline
			Men kan zijn financiële instelling kiezen en zo kiezen voor lage makelaarslonen & Roerende voorheffing op dividenden \\ \hline
			       & Kosten kunnen return gevoelig beïnvloeden  \\ \bottomrule
		\end{tabular}
	\caption{Voor-~en nadelen van aandelen.}
	\label{tab:voordelenennadelenaandelen}
\end{table}

\begin{comment}
\subsubsection{Voordelen}

\begin{enumerate}
  \item Stemrecht op de algemene vergadering
  \item Recht op een deel van de uitkeerbare winst
  \item Voorkeurrecht bij kapitaalverhoging
  \item Recht op informatie
  \item Recht op een deel van de liquidatiewaarde
\end{enumerate}

\todo[inline]{Nog de voor en nadelen opsommen van aandelen.}
\end{comment}

%-------------------------------------------------------------

%\section{Product 3: Kasbon of fonds of obligatie voor korte termijn}

\section{Product 3: Aandelenfonds KBC Equity Fund Industrials \& Infrastructure Classic}

%\todo[inline]{productfiche nog opzoeken en bijvoegen als bijlage en een referentie ernaar toe plaatsen in deze section}

Ik heb 36 aandelen in dit aandelenfonds gekocht ter waarde van 12468,24~\euro{} als aankoopwaarde in totaal.

Ik heb voor dit fonds gekozen omdat het een goed rendement heeft over verschillende jaren en het ook een iets minder hoog risico profiel heeft.

Het fonds heeft als instapkosten een percentage van 3,00\% en geen uitstapkosten als je je aandelen verkoopt van dit fonds. Het heeft wel 1,50\% als beheerskosten, wat wil zeggen dat de bank elk jaar met 1,50\% gaat lopen als werkingskosten.

Het kbc infrastructure zal het de komende jaren ook goed doen, doordat Donald Trump gezegd heeft om veel overheidsgeld te investeren in infrastructuur-werken in Amerika.

%\todo[inline]{nog veiligheid, liquiditeit, return, fiscaliteit en kosten bespreken en toepassen.}
\begin{figure}[tbph!]
	\centering
	%\includegraphics[width=18cm]{images/producten/overzichtsitetijdbefondsenenaandelen}
	\caption[Aandelen Proximus]{Aandelen Proximus gekocht op 19 december 2016}
	\label{fig:overzichtsitetijdbefondsenenaandelen}
\end{figure}
\newpage
De veiligheid is niet zo goed bij fondsen. Door de verschillende beoordelingen, portefeuillerisico's, risico's door soorten fonds, operationele risico's, kostenrisico's en kwalitatieve ratings kan een fonds waarde verliezen of minder populair worden waardoor het in waarde verliest.

Er is geen enkel probleem voor de liquiditeit bij dit fonds omdat het een beleggingsfonds met veranderlijk kapitaal (bevek) is. Die wordt namelijk gegarandeerd door de uitgevende instelling. Dat wil natuurlijk zeggen dat ik de op dat ogenblik gangbare prijs moet aanvaarden om het fonds te kunnen kopen of verkopen.

Voor de return wordt een dividend uitgekeerd. Hierbij speelt het type fonds (distributie of kapitalisatie) en de soort (aandelen hier) een belangrijke rol voor de te realiseren return.

Voor de fiscaliteit is er een beurstaks verschuldigd bij de verhandeling van beleggingsfondsen. Ook is er een belasting op meerwaarde idien de netto intrinsieke waarde (NIW) van het fonds op het moment van verkoop hoger is dan de netto intrinsieke waarde (NIW) op het moment van aankoop.

Ook zijn er nog extra kosten, zoals de instap- en uitstapkosten die afhangen van het soort fonds. Ook moet het bewaarloon betaald worden aan de financiële instelling die de bezittingen van het fonds onder bewaring heeft. De kosten voor de administratief agent worden aangerekend voor de boekhouding van het fonds en het berekenen van de waarde van het fonds.

%Altijd onderbouwen bij elk product:
%\begin{enumerate}
	%\item \textbf{waarom} heb je dit product gekozen?
	%\item wat is de \textbf{veiligheid} van dit product?
	%\item wat is de \textbf{liquiditeit} van dit product?
	%\item wat is de \textbf{return} van dit product?
	%\item wat is de \textbf{fiscaliteit} van dit product?
	%\item wat zijn de \textbf{kosten} van dit product?
%\end{enumerate}

\subsection{Conclusie}
%Wat zijn de voor- en nadelen van dit product?

%\todo[inline]{voor-en nadelen nog bespreken van een fonds.}

\begin{table}[!htbp]
	\centering
%	\begin{tabular}{@{}lr@{}}
		\begin{tabular}{c|c}
			\toprule
			
			 \cellcolor{lightgray}            +              &  \cellcolor{lightgray}                                      -                                        \\ \midrule
			 Risicospreiding   & Beheerkosten kunnen oplopen          \\ \hline
			  Professioneel beheer  & Roerende voorheffing op dividenden en obligatiefondsen \\ \hline
			Controle door FSMA & Kapitaalverlies mogelijk \\ \hline
			       Fiscale voordelen &   \\ \hline
			       Keuze volgens beleggersprofiel &   \\ \bottomrule
		\end{tabular}
	\caption{Voor-~en nadelen van een aandelenfonds.}
	\label{tab:voordelenennadelenaandelenfonds1}
\end{table}

%-------------------------------------------------------------

%\section{Product 4: Sicav of bevek in het nederlands}
\section{Product 4: technologie gerichte aandelenfonds Candriam Equities B Robotics \& Innovative Technology C}

%\todo[inline]{productfiche nog opzoeken en bijvoegen als bijlage en een referentie ernaar toe plaatsen in deze section}

Ik heb 73 aandelen in dit aandelenfonds gekocht ter waarde van 11908,64~\euro{} als aankoopwaarde in totaal.

Ik heb voor dit fonds gekozen omdat het een goed rendement heeft over verschillende jaren. Het fonds heeft dan wwel een hoog risico profiel, maar het is wel veel gestegen de laatste 4 jaren, dit betekent dus dat dit fonds goed beheerd wordt.

Het fonds heeft als instapkosten een percentage van 2,50\% en geen uitstapkosten als je je aandelen verkoopt van dit fonds. Het heeft wel 1,50\% als beheerskosten, wat wil zeggen dat de bank elk jaar met 1,50\% gaat lopen als werkingskosten.

%\todo[inline]{nog veiligheid, liquiditeit, return, fiscaliteit en kosten bespreken en toepassen.}

\begin{figure}[tbph!]
	\centering
%	\includegraphics[width=18cm]{images/producten/overzichtsitetijdbefondsenenaandelen}
	\caption[Aandelen Proximus]{Aandelen Proximus gekocht op 19 december 2016}
	\label{fig:overzichtsitetijdbefondsenenaandelen}
\end{figure}

De veiligheid is niet zo goed bij fondsen. Door de verschillende beoordelingen, portefeuillerisico's, risico's door soorten fonds, operationele risico's, kostenrisico's en kwalitatieve ratings kan een fonds waarde verliezen of minder populair worden waardoor het in waarde verliest.

Er is geen enkel probleem voor de liquiditeit bij dit fonds omdat het een beleggingsfonds met veranderlijk kapitaal (bevek) is. Die wordt namelijk gegarandeerd door de uitgevende instelling. Dat wil natuurlijk zeggen dat ik de op dat ogenblik gangbare prijs moet aanvaarden om het fonds te kunnen kopen of verkopen.

Voor de return wordt een dividend uitgekeerd. Hierbij speelt het type fonds (distributie of kapitalisatie) en de soort (aandelen hier) een belangrijke rol voor de te realiseren return.

Voor de fiscaliteit is er een beurstaks verschuldigd bij de verhandeling van beleggingsfondsen. Ook is er een belasting op meerwaarde idien de netto intrinsieke waarde (NIW) van het fonds op het moment van verkoop hoger is dan de netto intrinsieke waarde (NIW) op het moment van aankoop.

Ook zijn er nog extra kosten, zoals de instap- en uitstapkosten die afhangen van het soort fonds. Ook moet het bewaarloon betaald worden aan de financiële instelling die de bezittingen van het fonds onder bewaring heeft. De kosten voor de administratief agent worden aangerekend voor de boekhouding van het fonds en het berekenen van de waarde van het fonds.

%Altijd onderbouwen bij elk product:
%\begin{enumerate}
	%\item \textbf{waarom} heb je dit product gekozen?
	%\item wat is de \textbf{veiligheid} van dit product?
	%\item wat is de \textbf{liquiditeit} van dit product?
	%\item wat is de \textbf{return} van dit product?
	%\item wat is de \textbf{fiscaliteit} van dit product?
	%\item wat zijn de \textbf{kosten} van dit product?
%\end{enumerate}

\subsection{Conclusie}
%Wat zijn de voor- en nadelen van dit product?

%\todo[inline]{voor-en nadelen nog bespreken van een fonds.}

\begin{table}[!htbp]
	\centering
%	\begin{tabular}{@{}lr@{}}
		\begin{tabular}{c|c}
			\toprule
			
			 \cellcolor{lightgray}            +              &  \cellcolor{lightgray}                                      -                                        \\ \midrule
			 Risicospreiding   & Beheerkosten kunnen oplopen          \\ \hline
			  Professioneel beheer  & Roerende voorheffing op dividenden en obligatiefondsen \\ \hline
			Controle door FSMA & Kapitaalverlies mogelijk \\ \hline
			       Fiscale voordelen &   \\ \hline
			       Keuze volgens beleggersprofiel &   \\ \bottomrule
		\end{tabular}
	\caption{Voor-~en nadelen van een aandelenfonds.}
	\label{tab:voordelenennadelenaandelenfonds1}
\end{table}

%-------------------------------------------------------------

%%% Local Variables:
%%% mode: latex
%%% TeX-master: "masterproef"
%%% End:
