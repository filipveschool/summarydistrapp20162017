\newpage
\section{Spaarvermogen}\index{Spaarvermogen}

%Uitleg: Bereken je \textbf{spaarvermogen} per jaar. Bespreek kort hoe je dit zal beleggen, bereken je jaarlijkse return en voeg dit bij aan de inkomsten van het jaar erop.\newline\newline

%\todo[inline]{Uitleg hierboven op het einde nog verwijderen.}

Voor mijn doelstellingen heb ik 375000~\euro{} nodig om ze allemaal te kunnen betalen over een periode van maximaal 40 jaar.
Volgens de webtool van \fnurl{tijd.be}{{http://www.tijd.be/netto/tool/179}} moet ik dan maandelijks 733,38~\euro{} sparen.
 %Hierdoor zou ik 733,38~\euro{} per maand moeten sparen volgens mijn berekeningen die ik gemaakt heb met de webtool van tijd.be om te berekenen~\fnurl{"hoeveel je maandelijks moet sparen voor een eindkapitaal"}{http://www.tijd.be/netto/tool/179}.

\begin{figure}[!htbp]
	\centering
%	\includegraphics[width=14cm]{spaarvermogentijdbemaandelijkssparen.PNG}
	\caption{Sparen voor doelstellingen}
	\label{fig:sparendoelstellinen}
\end{figure}

Met mijn berekend spaarvermogen van 1412~\euro{} per maand en met men maandelijks spaarbedrag van 733,38~\euro{} (zie figuur

\todo[inline]{referentie naar de figuur linken nog oplossen.}

\ref{fig:sparendoelstellingen}) zijn mijn doelstellingen zeker realiseerbaar. De eerste jaren woon ik ook thuis en heb ik minder kosten dan wanneer ik alleen zou wonen. Zo zou ik dus ook meer op een spaarrekening kunnen zetten en zo meer profijt uit mijn geld halen.

Ik ben dan ook van plan om twee aparte spaarrekeningen te openen (namelijk de Deutsche Bank Saving Plan en Ing Tempo Sparen), zodat ik op beide spaarrekeningen maandelijks 500~\euro{} zou kunnen storten en zo meer geld aan een hogere rente laten opbrengen op een termijn van 5 jaar. Na die 5 jaar kan ik mijn geld altijd verplaatsen naar een andere spaarrekening moest er tegen dan iets beter en/of voordeliger zijn qua opbrengst en rente voor mijn geld.

Als ik dus via de twee aparte spaarrekeningen zou gaan en op elke spaarrekening 5000 euro van mijn huidige 13450~\euro{} zou storten als startbedrag, behaal ik na 5 jaar een totaal van 72803,83~\euro{} (zie vergelijking \ref{fig:tweespaarrekeningenoverzicht}). Maar als ik die 10000~\euro{} op mijn gewone spaarrekening zou laten gestaan hebben (dit is ter vergelijking en ter verduidelijking van het verschil), zou mijn 10000~\euro{} na 5 jaar 70223,20 geworden zijn. Dit is een verschil van 2580,60~\euro{}.
De rest van mijn startbedrag op mijn spaarrekening zou ik gewoon laten staan en ook de resterende 412~\euro{} (komt van het spaarvermogen 1412~\euro{} - 1000~\euro{} die naar de spaarrekening(en) gaat) wordt niet in rekening genomen omdat deze soms gestort zou kunnen worden of soms apart gehouden kan worden voor onverwachte zaken of cadeau's die je moet kopen voor familie en dat zijn kosten waar ik niet echt rekening mee kan houden op het moment.

Doordat ik dan 1000~\euro{} per maand sparen in rekening gehouden heb, zit ik ook ongeveer 250~\euro{} boven het berekende maandelijks spaarbedrag en bouw ik dus een goeie reserve op voor de toekomst. Na 5 jaar sparen zou mijn doelstelling om mijn eigen auto te kopen dus zeker moeten lukken.

Het is te verwachten dat de volgende 5 jaar de salarissen lichtjes zullen stijgen, waardoor de kans op het bereiken van mijn bedragen en doelstellingen verbeteren.

\begin{comment}

\begin{figure}[!htbp]
	\centering
	\includegraphics[width=14cm]{spaarvermogen/bedragsparendeutschebank.PNG}
	\caption{sparen voor doelstellingen}
	\label{fig:bedragsparendeutschebank}
\end{figure}

\begin{figure}[!htbp]
	\centering
	\includegraphics[width=14cm]{spaarvermogen/bedragspareningbank.PNG}
	\caption{sparen voor doelstellingen}
	\label{fig:bedragspareningbank}
\end{figure}

\end{comment}

\begin{figure}
    \centering
    \begin{subfigure}[b]{0.49\textwidth}        
    	%\includegraphics[width=\textwidth]{spaarvermogen/bedragsparendeutschebank.PNG}
        \caption{Sparen via Deutsche Bank saving plan}
        \label{fig:bedragsparendeutschebank}
    \end{subfigure}
    \begin{subfigure}[b]{0.49\textwidth}
%        \includegraphics[width=\textwidth]{spaarvermogen/bedragspareningbank.PNG}
        \caption{Sparen via ING Tempo Sparen}
        \label{fig:bedragspareningbank}
    \end{subfigure}
    \caption{10000 euro is 72803,83~\euro{} na 5 jaar geworden}\label{fig:tweespaarrekeningenoverzicht}
\end{figure}


\begin{figure}[!htbp]
	\centering
	%\includegraphics[width=14cm]{spaarvermogen/bedragsparenlagerente.PNG}
	\caption{10000 euro is 70223,20~\euro{} geworden na 5 jaar}
	\label{fig:bedragsparenlagerente}
\end{figure}



\begin{comment}
\begin{figure}
    \centering
    \begin{subfigure}[b]{0.3\textwidth}
        \includegraphics[width=\textwidth]{spaarvermogen/bedragsparendeutschebank.PNG}
        \caption{A gull}
        \label{fig:gull}
    \end{subfigure}
    \begin{subfigure}[b]{0.3\textwidth}
        \includegraphics[width=\textwidth]{spaarvermogen/bedragsparendeutschebank.PNG}
        \caption{A tiger}
        \label{fig:tiger}
    \end{subfigure}
    \caption{Pictures of animals}\label{fig:animals}
\end{figure}
\end{comment}




%\todo[inline]{Best eerst spaarbedrag berekenen dat nodig zou zijn voor je doelstellingen te kunnen bekostigen. Hier dan gebruik van maken voor "bedrag per maand" te sparen te berekenen via de tools op netto.tijd.be. }

\begin{comment}
Ik ga 10000 euro sparen over een periode van 10 jaar. Ik heb dit berekend met \fnurl{de spaarsimulator op spaargids.be}{http://www.spaargids.be/sparen/spaarsimulator.html} en ook via \fnurl{de kbc spaarsimulator}{https://www.kbc.be/PBL/CC028/spaarsimulator} en via \fnurl{de ING spaarsimulator}{https://www.ing.be/nl/retail/savings-calculator}.

\begin{figure}[!htbp]
	\centering
	\includegraphics[width=14cm]{kbcspaarsimulator.PNG}
	\caption{Kbc spaarsimulator}
	\label{fig:Kbc spaarsimulator}
\end{figure}

\begin{figure}[tbph!]
	\centering
	\includegraphics[width=14cm]{images/spaarsimulatorDeTijdBeTienJaarSparen}
	\caption[Dit is sparen voor 10 jaar]{Tien jaar sparen met 10000 \euro{} als startkapitaal.}
	\label{fig:spaarsimulatordetijd}
\end{figure}

\newpage
\subsection{netto.tijd.be tool gebruikt}

Dit is een voorbeeld van de docent. Dit kan ik gebruiken voor mezelf dan.



\begin{figure}[tbph!]
	\centering
	\includegraphics[width=14cm]{images/voorbeeld_van_de_tijd_be_netto_tools}
	\caption[Dit is een voorbeeld voor spaarvermogen]{Je gaat naar http://netto.tijd.be/tools. Dan onder categorie Sparen ga je naar "Hoeveel levert uw spaarinspanning op?".}
	\label{fig:voorbeeld-van-de-tijd}
\end{figure}
\end{comment}
