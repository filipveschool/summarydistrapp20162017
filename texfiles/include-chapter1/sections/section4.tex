\newpage
\section{Vermogensbalans na 5 jaar}
\index{Vermogensbalans! na 5 jaar}
Stel je \textbf{vermogensbalans} op na 5 jaar, dus op datum van 01 juli 2022. Gebruik het voorbeeld dat je kan vinden in het handboek beleggingsleer pagina 119.

%\todo[inline]{Alles aanpassingen in deze section voor de vermogensbalans binnen 5 jaar.}

\todo[inline]{Uitleg hierboven op het einde nog verwijderen.}

\subsection{Bezittingen}\index{Bezittingen}


%\begin{table}[!htbp]
	\centering
	\begin{tabular}{@{}lr@{}} \toprule
		\multicolumn{2}{c}{Beroepsinkomen} \\ \cmidrule(r){1-2}
		Inkomen    										& Euro (\euro{})\\ \midrule
		Beroepsinkomen man      						& 1696 \euro{} \\
		Beroepsinkomen vrouw    						& 0 \euro{} \\
		Maaltijdcheques       							& 307 \euro{} \\
		Dertiende maand en vakantiegeld  				& 0 \euro{} \\
		Bonussen 										& 0 \euro{} \\
		Inkomsten uit een zelfstandige activiteit		& 0 \euro{} \\
		\toprule
		\textbf{Totaal van de inkomsten}              	& \textbf{148,47 \euro{}}   \\ \bottomrule
	\end{tabular}
	\caption{Tabel van het beroepsinkomen}
	\label{tab:tabelberoepsinkomen}
\end{table}





%\subimport{../tabels/}{tabel7.tex}
%\subimport{../tabels/}{tabel8.tex}
%\subimport{../tabels/}{tabel9.tex} % dit is het volledige overzicht zoals in excel te zien was.

Aangezien ik geen onroerende en zakelijke bezittingen\index{Bezittingen!onroerende en zakelijke bezittingen} heb, doordat ik nog thuis woon en pas begin te werken, heb ik dus nog geen bezittingen. Ook heb ik geen verzekering, behalve die van het bedrijf.


%\vspace{-1em}
\begin{table}[ht]
	%	\begin{minipage}[b]{0.45\linewidth}\centering
	\begin{minipage}[b]{9cm}\centering
		\begin{tabular}{@{}lr@{}}
			\toprule
			\multicolumn{2}{c}{Onroerende en zakelijke bezittingen}  \\
			\cmidrule(r){1-2}
			Bezittingen &        Euro (\euro{}) \\ \midrule
			Woning                           &             0 \euro{} \\
			Tweede woning                    &             0 \euro{} \\
			Eigen zaak                       &             0 \euro{} \\
			Verhuurd vastgoed                &             0 \euro{} \\
			\dots                            &               \euro{} \\ \toprule
			\textbf{Totaal van de bezittingen} & \textbf{0 \euro{}} \\ \bottomrule
		\end{tabular}
		\caption{Tabel van de onroerende en zakelijke bezittingen}
		\label{tab:tabelonroerendeenzakelijkebezittingen}
		
		
	\end{minipage}
	\hspace{0.5cm}
	%	\begin{minipage}[b]{0.45\linewidth}\centering
	\begin{minipage}[b]{9cm}\centering
		\begin{tabular}{@{}lr@{}}
			\toprule
			\multicolumn{2}{c}{Verzekerings- en pensioenbeleggingen}   \\
			\cmidrule(r){1-2}
			Bezittingen      &      Euro (\euro{}) \\ \midrule
			Pensioensparen                        &           0 \euro{} \\
			Privé-levensverzekering               &           0 \euro{} \\
			Groeps- of bedrijfsleidersverzekering &           0 \euro{} \\
			\dots                                 &          0 \euro{} \\ \toprule
			\textbf{Totaal van de bezittingen}    & \textbf{0 \euro{}} \\ \bottomrule
		\end{tabular}
		\caption{Tabel van de verzekerings- en pensioenbeleggingen}
		\label{tab:tabelverzekeringsenpensioenbeleggingen}
	\end{minipage}
\end{table}
%\vspace{-1em}

Aangezien ik veronderstel dat mijn oude wagen na 5 jaar niet meer zal rijden, bedraagt mijn persoonlijke bezittingen hiervoor 0~\euro{}. Ondertussen heb ik 5 jaar gespaard om mijn eigen wagen te kunnen kopen.


%\vspace{-1em}
\begin{table}[ht]
	%	\begin{minipage}[b]{0.45\linewidth}\centering
	\begin{minipage}[b]{9cm}\centering
		\begin{tabular}{@{}lr@{}}
			\toprule
			\multicolumn{2}{c}{Persoonlijke bezittingen}       \\
			\cmidrule(r){1-2}
			Bezittingen   &     Euro (\euro{}) \\ \midrule
			Auto's                             &          0 \euro{} \\
			Uitrusting van de woning           &          0 \euro{} \\
			Kunstvoorwerpen                    &          0 \euro{} \\
			\dots                              &            \euro{} \\ \toprule
			\textbf{Totaal van de bezittingen} & \textbf{0 \euro{}} \\ \bottomrule
		\end{tabular}
		\caption{Tabel van de onroerende en zakelijke bezittingen}
		\label{tab:tabelonroerendeenzakelijkebezittingen}
		
		
	\end{minipage}
	\hspace{0.5cm}
	%	\begin{minipage}[b]{0.45\linewidth}\centering
	\begin{minipage}[b]{9cm}\centering
		\begin{tabular}{@{}lr@{}}
			\toprule
			\multicolumn{2}{c}{Beleggingsportefeuille}        \\
			\cmidrule(r){1-2}
			Bezittingen   &      Euro (\euro{}) \\ \midrule
			
			Spaarrekening                      &           102.067,61 \euro{} \\
			Termijnrekening, kasbon, staatsbon &           0 \euro{} \\
			Obligaties                         &           0 \euro{} \\
			Aandelen                           &           0 \euro{} \\
			Beleggingsfondsen                  &           0 \euro{} \\
			\dots                              &  \\ \toprule
			\textbf{Totaal van de bezittingen} & \textbf{13082 \euro{}} \\ \bottomrule
		\end{tabular}
		\caption{Tabel van de beleggingsportefeuille}
		\label{tab:tabelbeleggingsportefeuille}
	\end{minipage}
\end{table}
%\vspace{-1em}
% spaarrekening starten met 10682 + (7 * 50 euro per maand van de ouders) + 1000 van bobonne voor kerstmis + (7 * 150 per maand van bobonne qua storting) = 13082 euro
%
%

Ik heb ook de resterende 412~\euro{} van mijn netto spaarvermogen (1412-500-500) als maandelijks spaarbedrag voor mijn huidige spaarrekening. 
%Dit is puur theoretisch om de bedragen te kunnen berekenen. 
Hierdoor heb ik twee spaarrekeningen, met een hoger intrestpercentage(1,50\% en 1,20\%), die dan samen geteld worden bij mijn lopende KBC-rekening: totaal 102067,61~\euro{}. 
%Ik zou deze dan ook indien mogelijk en als het voordelig zijn voor mezelf door genoeg kilometers af te leggen per jaar, kunnen inbrengen in mijn belastingen en hierdoor kosten recupereren van mijn belastingen.

\begin{figure}[!htbp]
	\centering
	%\includegraphics[width=14cm]{spaarvermogen/bedragsparenhuidigespaarrekening.PNG}
	\caption{Mijn huidige spaarrekening bij KBC}
	\label{fig:bedragsparenhuidigespaarrekening}
\end{figure}


\newpage
\subsection{Schulden}\index{Schulden}

%\todo[inline]{Hoe moet dit geformuleerd worden? Wat is er al mogelijk voor ons? Moet dit per maand of per jaar vermeld worden qua prijs in de tabel?}


Momenteel woon ik nog steeds thuis waardoor ik nog geen huis of appartement heb gekocht. Hierdoor heb ik nog geen hypothecaire lening lopen. Ik heb ook nog geen tv of meubels gekocht op consumentenkrediet en ga geen geld lenen bij familie. Hierdoor bedraagt alles 0~\euro{} staan.

\begin{table}[ht]
	\begin{minipage}[b]{0.45\linewidth}\centering
		%		\begin{table}[!htbp]
		%			\centering
		\begin{tabular}{@{}lr@{}}
			\toprule
			\multicolumn{2}{c}{Leningen voor onroerende en zakelijke bezittingen} \\
			\cmidrule(r){1-2}
			Schulden   &                      Euro (\euro{}) \\ \midrule
			Hypothecaire lening             &                           0 \euro{} \\
			Investeringskrediet             &                         0 \euro{} \\
			\dots                           &                             \euro{} \\ \toprule
			\textbf{Totaal van de schulden} &               \textbf{0 \euro{}} \\ \bottomrule
		\end{tabular}
		\caption{Tabel van de leningen voor onroerende en zakelijke bezittingen}
		\label{tab:tabelleningenvooronroerendeenzakelijkebezittingen}
		%		\end{table}
		
		
	\end{minipage}
	\hspace{2cm}
	\begin{minipage}[b]{0.45\linewidth}
		\centering
		%		\begin{table}[!htbp]
		%			\centering
		\begin{tabular}{@{}lr@{}}
			\toprule
			\multicolumn{2}{c}{Leningen voor persoonlijke bezittingen} \\
			\cmidrule(r){1-2}
			Schulden   &           Euro (\euro{}) \\ \midrule
			Consumentenkrediet              &                0 \euro{} \\
			\dots                           &  \\ \toprule
			\textbf{Totaal van de schulden} &       \textbf{0 \euro{}} \\ \bottomrule
		\end{tabular}
		\caption{Tabel van de leningen voor persoonlijke bezittingen}
		\label{tab:tabelleningenvoorpersoonlijkebezittingen}
		%		\end{table}
	\end{minipage}
\end{table}

\begin{table}[!htbp]
	\centering
	\begin{tabular}{@{}lr@{}}
		\toprule
		                \multicolumn{2}{c}{Andere schulden}                 \\
		\cmidrule(r){1-2}
		Schulden                   &     Euro (\euro{}) \\ \midrule
		Vb. geld geleend van familie, kennissen, \dots &          0 \euro{} \\
		\dots                                          &  \\ \toprule
		\textbf{Totaal van de schulden}                & \textbf{0 \euro{}} \\ \bottomrule
	\end{tabular}
	\caption{Tabel van andere schulden}
	\label{tab:tabelandereschulden}
\end{table}

Het totaal nettovermogen van mijn vermogensbalans op 1 juli 2022 bedraagt 102067,61~\euro{} (= bezittingen: 102067,61~\euro{} - schulden: 0~\euro{}).

%Hiermee kan ik dan mijn auto kopen, investeringen doen of dergelijke. Mogelijks zijn er in die 5 jaar misschien veranderingen doorgevoerd waar ik nu nog niet van weet zoals eventueel samenwonen als ik niet meer vrijgezel ben of gewoon alleen gaan wonen. Ik heb gewoon berekend dat ik op dit moment geen schulden heb na de 5 jaar.

