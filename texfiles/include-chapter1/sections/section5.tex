\newpage
\section{Doelstellingen}\index{Doelstellingen}

%Wat zijn de \textbf{doelstellingen} op korte, middellange en lange termijn? Wil je sparen voor een auto of een grote reis? Wil je binnen een aantal jaren een woning of appartement kopen? Als je verschillende doelstellingen hebt, welke zijn dan het belangrijkste? Welke bedragen heb je hiervoor nodig? Maak een actieplan op om deze doelstelling te halen.
%\todo[inline]{inleiding tekst hierboven nog wegdoen op het einde!!!}



Ik wil voor een totaalbedrag van 375000 \euro{} sparen voor al mijn doelstellingen samen.
Hiervoor zou ik gebruik maken van verschillende spaarrekeningen met een hoog rentepercentage. Ik zou de eerste jaren alleen spaarrekeningen willen gebruiken omdat ik het in het begin nog op veilig wil houden.
Omdat het maximaal te storten bedraagt 500~\euro{} per maand, omdat het maximaal te storten bedrag 500~\euro{} per maand bedraagt, ga ik mijn stortingen verspreiden over meerdere verschillende spaarrekeningen. Spreiding van geld is ook beter in geval van faillissement van één of andere bank.
Voor de infofiche van de beide gebruikte spaarrekeningen zie bijlage: \autoref{appendix:infoficheingendeutschebanksparen}

\begin{comment}
\begin{figure}[!htbp]
  \centering
  \includegraphics[width=14cm]{images/ingtemposparenspaarrekening}
  \caption{Rentepercentage van 1,20\% gebruikt van ING Tempo Sparen spaarrekening }
  \label{fig:ingtemposparenspaarrekening}
\end{figure}

\begin{figure}[!htbp]
  \centering
  \includegraphics[width=14cm]{images/deutschebankdbsavingplanspaarrekening}
  \caption{Rentepercentage van 1,50\% gebruikt van ING Tempo Sparen spaarrekening}
  \label{fig:deutschebankdbsavingplanspaarrekening}
\end{figure}
\end{comment}

\begin{figure}
    \centering
    \begin{subfigure}[b]{0.49\textwidth}        
    	%\includegraphics[width=\textwidth]{images/ingtemposparenspaarrekening}
        \caption{Sparen via Deutsche Bank saving plan}
        \label{fig:bedragsparendeutschebank}
    \end{subfigure}
    \begin{subfigure}[b]{0.49\textwidth}
        %\includegraphics[width=\textwidth]{images/deutschebankdbsavingplanspaarrekening}
        \caption{Sparen via ING Tempo Sparen}
        \label{fig:bedragspareningbank}
    \end{subfigure}
    %\caption{10000 euro is 72803,83~\euro{} na 5 jaar geworden}
    \label{fig:tweespaarrekeningenoverzicht}
\end{figure}


%\todo[inline]{Images naast elkaar zetten!!!}



\newpage
\subsection{Doelstellingen op korte termijn}\index{Doelstellingen!op korte termijn}

\subsubsection{Doelstelling 1 : auto kopen}

%Voorbeeld: Ik heb 20000 \euro{} nodig voor een auto te kopen. Hiervoor ga ik 5 jaar sparen.

Ik wil op korte termijn een eigen auto kopen voor een bedrag van 25000~\euro{}. Hiervoor zou ik 5 jaar willen sparen.

%Hierdoor zou ik niet meer afhankelijk zijn van mijn bedrijf. Ook kan ik de auto dan afschrijven voor belastingsvoordelen. En indien ik ooit onverwacht ontslagen wordt, heb ik tenminste nog een wagen die ik kan gebruiken. Ik ga deze wagen afronden op een 25000 \euro{} als bedrag. Ik zou hiervoor 5jaar willen sparen.

\begin{figure}[!htbp]
	\centering
	%\includegraphics[width=14cm]{images/voorbeelddoelstellingsparen}
	\caption{Doelstelling sparen voor mijn eigen auto}
	\label{fig:voorbeelddoelstellingsparen}
\end{figure}


\subsection{Doelstellingen op middellange termijn}\index{Doelstellingen!op middellange termijn}

\subsubsection{Doelstelling 2: Studiekosten kinderen in de toekomst en financiële buffer}

Ik ben momenteel nog vrijgezel en wil een financiële buffer voorzien van 100000 \euro{} in de komende 20 jaar. Dit bedrag kan ook gebruikt worden voor de studiekosten van mogelijks toekomstige kinderen.

Indien ik niet meer vrijgezel ben na een paar jaar, zou dit bedrag kunnen verhoogd worden of zou de termijn verlaagd kunnen worden indien nodig.

\subsection{Doelstellingen op lange termijn}\index{Doelstellingen!op lange termijn}

\subsubsection{Doelstelling 3: Spaarfonds voor het pensioen}
Tegen de pensioenleeftijd zou ik graag 250000~\euro{} gespaard hebben zeker. Onder andere voor een grote wereldreis en om te kunnen genieten van een comfortable oudere dag zonder financiële zorgen.

%\lipsum[1-5] 