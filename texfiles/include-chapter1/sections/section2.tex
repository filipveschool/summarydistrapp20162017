\newpage
\section{Vermogensbalans}\index{Vermogensbalans}
\index{Vermogensbalans!De beginbalans}

\subsection{Bezittingen}\index{Bezittingen}

%\begin{table}[!htbp]
	\centering
	\begin{tabular}{@{}lr@{}} \toprule
		\multicolumn{2}{c}{Beroepsinkomen} \\ \cmidrule(r){1-2}
		Inkomen    										& Euro (\euro{})\\ \midrule
		Beroepsinkomen man      						& 1696 \euro{} \\
		Beroepsinkomen vrouw    						& 0 \euro{} \\
		Maaltijdcheques       							& 307 \euro{} \\
		Dertiende maand en vakantiegeld  				& 0 \euro{} \\
		Bonussen 										& 0 \euro{} \\
		Inkomsten uit een zelfstandige activiteit		& 0 \euro{} \\
		\toprule
		\textbf{Totaal van de inkomsten}              	& \textbf{148,47 \euro{}}   \\ \bottomrule
	\end{tabular}
	\caption{Tabel van het beroepsinkomen}
	\label{tab:tabelberoepsinkomen}
\end{table}





%\subimport{../tabels/}{tabel7.tex}
%\subimport{../tabels/}{tabel8.tex}
%\subimport{../tabels/}{tabel9.tex} % dit is het volledige overzicht zoals in excel te zien was.

Aangezien ik geen onroerende en zakelijke bezittingen\index{Bezittingen!onroerende en zakelijke bezittingen} heb, doordat ik nog thuis woon en pas begin te werken, heb ik dus nog geen bezittingen. Ook heb ik geen verzekering, behalve die van het bedrijf.

%\vspace{-1em}
\begin{table}[ht]
	%	\begin{minipage}[b]{0.45\linewidth}\centering
	\begin{minipage}[b]{9cm}\centering
		\begin{tabular}{@{}lr@{}}
			\toprule
			\multicolumn{2}{c}{Onroerende en zakelijke bezittingen}  \\
			\cmidrule(r){1-2}
			Bezittingen &        Euro (\euro{}) \\ \midrule
			Woning                           &             0 \euro{} \\
			Tweede woning                    &             0 \euro{} \\
			Eigen zaak                       &             0 \euro{} \\
			Verhuurd vastgoed                &             0 \euro{} \\
			\dots                            &               \euro{} \\ \toprule
			\textbf{Totaal van de bezittingen} & \textbf{0 \euro{}} \\ \bottomrule
		\end{tabular}
		\caption{Tabel van de onroerende en zakelijke bezittingen}
		\label{tab:tabelonroerendeenzakelijkebezittingen}
		
		
	\end{minipage}
	\hspace{0.5cm}
	%	\begin{minipage}[b]{0.45\linewidth}\centering
	\begin{minipage}[b]{9cm}\centering
		\begin{tabular}{@{}lr@{}}
			\toprule
			 \multicolumn{2}{c}{Verzekerings- en pensioenbeleggingen}   \\
			\cmidrule(r){1-2}
			Bezittingen      &      Euro (\euro{}) \\ \midrule
			Pensioensparen                        &           0 \euro{} \\
			Privé-levensverzekering               &           0 \euro{} \\
			Groeps- of bedrijfsleidersverzekering &           0 \euro{} \\
			\dots                                 &          0 \euro{} \\ \toprule
			\textbf{Totaal van de bezittingen}    & \textbf{0 \euro{}} \\ \bottomrule
		\end{tabular}
		\caption{Tabel van de verzekerings- en pensioenbeleggingen}
		\label{tab:tabelverzekeringsenpensioenbeleggingen}
	\end{minipage}
\end{table}
%\vspace{-1em}

Voor de persoonlijke bezittingen\index{Bezittingen!persoonlijke bezittingen} heb ik zelf een oude wagen die nog op 1000~\euro{} geschat wordt. Voor mijn beleggingsportefeuille heb ik berekend hoeveel er op mijn spaarrekening zal staan op 1 juli 2017 door de te verwachten maandelijkse stortingen bij te tellen bij het huidige bedrag. Dit is berekend door het huidige bedrag (10000\euro{}) samen te tellen met de jaarlijkse storting in januari van mijn oma, bij de maandelijkse stortingen tot 1 juli 2017. Dit maakt een bedrag van 3450~\euro{} over 7 maanden stortingen van de familie en de jaarlijkse storting bij nieuwjaar. Dit maakt een totaal van 13450~\euro{} op mijn spaarrekening op 1 juli 2017 als startbedrag.

%\vspace{-1em}
\begin{table}[ht]
	%	\begin{minipage}[b]{0.45\linewidth}\centering
	\begin{minipage}[b]{9cm}\centering
		\begin{tabular}{@{}lr@{}}
			\toprule
			     \multicolumn{2}{c}{Persoonlijke bezittingen}       \\
			\cmidrule(r){1-2}
			Bezittingen   &     Euro (\euro{}) \\ \midrule
			Auto's                             &          1000 \euro{} \\
			Uitrusting van de woning           &          0 \euro{} \\
			Kunstvoorwerpen                    &          0 \euro{} \\
			\dots                              &            \euro{} \\ \toprule
			\textbf{Totaal van de bezittingen} & \textbf{1000 \euro{}} \\ \bottomrule
		\end{tabular}
		\caption{Tabel van de onroerende en zakelijke bezittingen}
		\label{tab:tabelonroerendeenzakelijkebezittingen}
		
		
	\end{minipage}
	\hspace{0.5cm}
	%	\begin{minipage}[b]{0.45\linewidth}\centering
	\begin{minipage}[b]{9cm}\centering
		\begin{tabular}{@{}lr@{}}
			\toprule
			       \multicolumn{2}{c}{Beleggingsportefeuille}        \\
			\cmidrule(r){1-2}
			Bezittingen   &      Euro (\euro{}) \\ \midrule
		
			Spaarrekening                      &           13450 \euro{} \\
			Termijnrekening, kasbon, staatsbon &           0 \euro{} \\
			Obligaties                         &           0 \euro{} \\
			Aandelen                           &           0 \euro{} \\
			Beleggingsfondsen                  &           0 \euro{} \\
			\dots                              &  \\ \toprule
			\textbf{Totaal van de bezittingen} & \textbf{13450 \euro{}} \\ \bottomrule
		\end{tabular}
		\caption{Tabel van de beleggingsportefeuille}
		\label{tab:tabelbeleggingsportefeuille}
	\end{minipage}
\end{table}
%\vspace{-1em}
% spaarrekening starten met 10682 + (7 * 50 euro per maand van de ouders) + 1000 van bobonne voor kerstmis + (7 * 150 per maand van bobonne qua storting) = 13082 euro
%
%



\newpage
\subsection{Schulden}\index{Schulden}

%\todo[inline]{Hoe moet dit geformuleerd worden? Wat is er al mogelijk voor ons? Moet dit per maand of per jaar vermeld worden qua prijs in de tabel?}

Ik blijf de eerste jaren thuis wonen waardoor ik geen huis of appartement moet kopen. Hierdoor heb ik geen hypothecaire lening nodig. Ik ga ook geen tv of meubels kopen op consumentenkrediet en ga geen geld lenen bij familie. Hierdoor blijft alles op 0~\euro{} staan.

%\begin{table}[!htbp]
	\centering
	\begin{tabular}{@{}lr@{}} \toprule
		\multicolumn{2}{c}{Beroepsinkomen} \\ \cmidrule(r){1-2}
		Inkomen    										& Euro (\euro{})\\ \midrule
		Beroepsinkomen man      						& 1696 \euro{} \\
		Beroepsinkomen vrouw    						& 0 \euro{} \\
		Maaltijdcheques       							& 307 \euro{} \\
		Dertiende maand en vakantiegeld  				& 0 \euro{} \\
		Bonussen 										& 0 \euro{} \\
		Inkomsten uit een zelfstandige activiteit		& 0 \euro{} \\
		\toprule
		\textbf{Totaal van de inkomsten}              	& \textbf{148,47 \euro{}}   \\ \bottomrule
	\end{tabular}
	\caption{Tabel van het beroepsinkomen}
	\label{tab:tabelberoepsinkomen}
\end{table}





%\subimport{../tabels/}{tabel10.tex}
%\subimport{../tabels/}{tabel11.tex}
%\subimport{../tabels/}{tabel12.tex}

\begin{table}[ht]
	\begin{minipage}[b]{0.45\linewidth}\centering
		%		\begin{table}[!htbp]
		%			\centering
		\begin{tabular}{@{}lr@{}}
			\toprule
			\multicolumn{2}{c}{Leningen voor onroerende en zakelijke bezittingen} \\
			\cmidrule(r){1-2}
			Schulden   &                      Euro (\euro{}) \\ \midrule
			Hypothecaire lening             &                           0 \euro{} \\
			Investeringskrediet             &                         0 \euro{} \\
			\dots                           &                             \euro{} \\ \toprule
			\textbf{Totaal van de schulden} &               \textbf{0 \euro{}} \\ \bottomrule
		\end{tabular}
		\caption{Tabel van de leningen voor onroerende en zakelijke bezittingen}
		\label{tab:tabelleningenvooronroerendeenzakelijkebezittingen}
		%		\end{table}
		
		
	\end{minipage}
	\hspace{2cm}
	\begin{minipage}[b]{0.45\linewidth}
		\centering
		%		\begin{table}[!htbp]
		%			\centering
		\begin{tabular}{@{}lr@{}}
			\toprule
			\multicolumn{2}{c}{Leningen voor persoonlijke bezittingen} \\
			\cmidrule(r){1-2}
			Schulden   &           Euro (\euro{}) \\ \midrule
			Consumentenkrediet              &                0 \euro{} \\
			\dots                           &  \\ \toprule
			\textbf{Totaal van de schulden} &       \textbf{0 \euro{}} \\ \bottomrule
		\end{tabular}
		\caption{Tabel van de leningen voor persoonlijke bezittingen}
		\label{tab:tabelleningenvoorpersoonlijkebezittingen}
		%		\end{table}
	\end{minipage}
\end{table}

\begin{table}[!htbp]
	\centering
	\begin{tabular}{@{}lr@{}}
		\toprule
		                \multicolumn{2}{c}{Andere schulden}                 \\
		\cmidrule(r){1-2}
		Schulden                   &     Euro (\euro{}) \\ \midrule
		Vb. geld geleend van familie, kennissen, \dots &          0 \euro{} \\
		\dots                                          &  \\ \toprule
		\textbf{Totaal van de schulden}                & \textbf{0 \euro{}} \\ \bottomrule
	\end{tabular}
	\caption{Tabel van andere schulden}
	\label{tab:tabelandereschulden}
\end{table}
Het totaal nettovermogen voor mijn vermogensbalans op 1 juli 2017, krijg ik na het verschil te berekenen tussen het totaal van de bezittingen (14450~\euro{}) en het totaal van de schulden (0~\euro{}). Dit maakt een nettovermogen van \textbf{14450~\euro{}}.

\begin{comment}

\begingroup

%\setlength{\arrayrulewidth}{0.6mm}
%This sets the thickness of the borders of the table. In the example is 1mm but you can use other units, see the article Lengths in LaTeX for a complete list.
%\setlength{\tabcolsep}{18pt}
%The space between the text and the left/right border of its containing cell is set to 18pt with this command. Again, you may use other units if needed.
\renewcommand{\arraystretch}{1.5}
%The height of each row is set to 1.5 relative to its default height.

\begin{table}[!htbp]
	\centering
	%\begin{tabular}{|l|l|l|l|}
	\begin{tabular}{l l l l}
%		\begin{tabular}{p{6cm} l p{5cm} l}
		\arrayrulecolor{black}
		\hline
		%\toprule
		\rowcolor{purpleexcel}
		\multicolumn{4}{c}{\textcolor{white}{\textbf{BUDGET}}} \\ \hline
		
		\rowcolor{purpleexcel}
		\multicolumn{1}{c}{\textcolor{white}{\textbf{Inkomsten}}} & \multicolumn{1}{c}{\textcolor{white}{\textbf{euro}}} &
		\multicolumn{1}{c}{\textcolor{white}{\textbf{Uitgaven}}} & \multicolumn{1}{c}{\textcolor{white}{\textbf{euro}}} \\ \hline
		%\centering Inkomsten & \centering euro & \centering Uitgaven & \centering euro \tabularnewline \hline
		
		\textbf{1. Beroepsinkomen}                      & \textbf{1535 \euro{}} & \textbf{1. Maandelijks terugkerende uitgaven} & 1005 \euro{} \\ \hline
		Beroepsinkomen man                              & 1535 \euro{}          & Huishuur of afbetaling van de lening          & 0 \euro{} \\ \hline
		Beroepsinkomen vrouw                            & 0  \euro{}            & Uitgven voor voeding en kleding               & 700 \euro{} \\ \hline
		Maaltijdcheques                                 & 0 \euro{}             & Elektriciteit, water, gas, internet, ...      & 305 \euro{} \\ \hline
		Dertiende maand en vakantiegeld                 & 0 \euro{}             & ...                                           & 0 \euro{} \\ \hline
		Bonussen                                        & 0 \euro{}             &                                               &  \\ \hline
		Inkomsten uit zelfstandige activiteit           & 0 \euro{}             &                                               &  \\ \hline
		...                                             &                       &                                               &   \\ \hline
		&                       &                                               & \\ \hline
		
		\textbf{2. Andere inkomsten}                    & 0 \euro{}             & \textbf{2. Jaarlijks terugkerende uitgaven}   & 970 \euro{} \\ \hline
		kindergeld                                      & 0 \euro{}             & Brand- en autoverzekering                     & 720 \euro{} \\ \hline
		Huurinkomsten                                   & 0 \euro{}             & Onderhoud auto                                & 250 \euro{} \\ \hline
		Teruggave van belastingen                       & 0 \euro{}             & Onderhoud woning                              & 0 \euro{} \\ \hline
		Intresten, divendenden van beleggingen, ...     & 0 \euro{}             & Studies kinderen                              & 0 \euro{} \\ \hline
		&                       & Belastingen                                   & 0 \euro{} \\ \hline
		&                       & Vakantie                                      & 0 \euro{} \\ \hline
		&                       & Diversen                                      & 0 \euro{} \\ \hline
		&                       & ...                                           & 0 \euro{} \\ \hline
		&                       &                                               &   \\ \hline
		&                       &                                               &   \\ \hline
		&                       &                                               &   \\ \hline
		\textbf{Totaal van de inkomsten}              & \textbf{1535 \euro{}} & \textbf{Totaal van de uitgaven \euro{}}       & \textbf{1975} \\ %\hline
		%                                        &               &                                               &       \\
		%\multicolumn{4}{|c|}{} \\ %\hline
		%\multicolumn{4}{|c|}{} \\ %\hline
		%Totaal van de inkomsten & 1535
		\bottomrule
		\rowcolor{red}
		\multicolumn{4}{c}{\textcolor{white}{\textbf{Spaarvermogen = -440 \euro{}}}} \\ \hline
		
	\end{tabular}
	\caption{Vermogensbalans}
	\label{tab:vermogensbalans}
\end{table}

\endgroup

\end{comment} 