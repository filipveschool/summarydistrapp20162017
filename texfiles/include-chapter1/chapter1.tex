\chapter{Financieel plan}
\label{Maven}
\thispagestyle{chapternohead}

	
\pagestyle{ruledfilip}
\subimport{sections/}{section1.tex}
%\newpage
\section{Explain that Maven\index{Maven} is easy to use because of ‘convention over Configuration’}

\begin{itemize}
	\item De programmeur moet minder beslissingen maken over de locatie van sommige zaken. Er is een standaard afgesproken structuur die Maven altijd gaat gebruiken en die je dus zelf niet moet specifieren. (Locatie libraries, packages en klassen)
	\item Pom.xml erft over van Super Pom waar al standaard heel wat in staat geconfigureerd
\end{itemize}

\section{Explain how Maven helps you with dependency management}
\begin{enumerate}
	\item Maven leest in de pom.xml welke jars nodig zijn
	\item Maven controleert of deze jars al in de local repository zitten
	\item Zo niet, download Maven deze jars van de remote repository (Web Server)
	\item Maven voegt deze jars toe aan de local repository
\end{enumerate}

$\Rightarrow$ Dit door een build uit te voeren. Het install commando wordt hier gebruikt

\section{Know that the build lifecycle has different phases and that each phase calls the previous one}

\begin{itemize}
	\item Validate = Valideer het project + controleer of alle noodzakelijke info aanwezig is
	\item Compile = Compileren van de source code
	\item Test = Test de gecompileerde code door gebruik van een testing framework
	\item Package =  Gecompileerde code vormen tot gedistribueerd formaat (JAR, WAR, ...)
	\item Integration-test = Package deployen naar een testomgeving en hierop tests uitvoeren
	\item Verify = Verifiëren van de package + bepaalde criteria controleren
	\item Install = Installeren van de package in de local repository
	\item Deploy = Kopieëren van de package naar de remote repository (Vb: Glassfish)
\end{itemize}

$\Rightarrow$ Als één fase wordt uitgevoerd, worden alle fases daarvoor ook uitgevoerd!

Extra info: \url{http://maven.apache.org/guides/introduction/introduction-to-the-lifecycle.html}

\section{Know that tasks are executed by plugins.}
\begin{itemize}
	\item De nodige plugins in de <build> tag (pom.xml) plaatsen
	\item Deze plugins zullen uitgevoerd worden na een build
\end{itemize}

$\Rightarrow$ Het doel van een plugin: Een specifieke taak, kleiner dan een fase, uitvoeren

\section{Know that Maven has a number of standard plugins, which are called automatically when you execute a standard build.}

Welke zijn die standaard plugins, die automatisch worden uitgevoerd???

Volledige lijst van plugins: \url{http://maven.apache.org/plugins/}

\section{Know that you can add plugins for non-standard tasks.}

\begin{itemize}
	\item glassfish:start-domain/stop-domain: Het domein (her)starten
	\item glassfish:deploy/redeploy: Programma’s deployen naar de (lokale) glassfish server
\end{itemize}

$\Rightarrow$ Hiervoor moet men de Glassfish plugin(s) toevoegen!

\section{Know that you can write your own plugins.}


Voorbeeld:

\begin{minted}{java}

package sample.plugin;
import org.apache.maven.plugin.AbstractMojo;
import org.apache.maven.plugin.MojoExecutionException;
import org.apache.maven.plugins.annotations.Mojo;

/*
*
* Says "Hi" to the user.
*
*/
@Mojo( name = "sayhi")
public class GreetingMojo extends AbstractMojo
{
public void execute() throws MojoExecutionException
{
getLog().info( "Hello, world." );
}
}
\end{minted}

Ik denk niet dat we zo iets gaan moeten implementeren, nooit in de les/opdracht gebeurd ofwel?

Uitleg hoe dit te doen: 
\url{http://maven.apache.org/guides/plugin/guide-java-plugin-development.html}

\section{Explain the steps in the picture of the Repository.}

%text webadres
%\fnurl{Het loon junior}{http://www.jobat.be/nl/artikels/het-loon-van}

\subsection{Inkomsten}\index{Inkomsten}

De inkomsten bestaan uit twee delen: het beroepsinkomen\index{Inkomsten!beroepsinkomen} en de andere inkomsten\index{Inkomsten!andere inkomsten}.
Het beroepsinkomen bestaat uit verschillende onderdelen. Ten eerste is er het beroepsinkomen van de man (1557~\euro{} nettoloon en 2125~\euro{} brutoloon). Het beroepsinkomen van de vrouw is 0~\euro{} omdat ik vrijgezel ben. De maaltijdcheques zijn berekend op 220 gewerkte dagen te vermenigvuldigen met de waarde van een maaltijdcheque, namelijk 8~\euro{}. Dus hebben we een totaal van 1760~\euro{}.

Het kindergeld is uiteraard 0~\euro{} aangezien ik vrijgezel ben en geen kinderen heb. De huurinkomsten zijn ook 0~\euro{} aangezien ik nog geen eigendommen te verhuren heb en omdat het het eerste jaar is dat ik werk, kan ik ook niks terugkrijgen van de belastingen.
Dit alles maakt dat ik een totaal heb qua inkomsten van 20444~\euro{} per jaar of 1557~\euro{} per maand.

%\todo[inline]{Maaltijdcheques berekenen. Hoe?}


%--------------------------------------------------------
%--------------------------------------------------------
%--------------------------------------------------------
\import{sections/include-chapter1/sections/}{section2.tex}
%\subimport{sections/}{section2.tex}

%--------------------------------------------------------
%--------------------------------------------------------
%--------------------------------------------------------

\subimport{sections/}{section3.tex}

%--------------------------------------------------------
%--------------------------------------------------------
%--------------------------------------------------------

% Reset the margins to be symmetric
%\setlrmarginsandblock{3.5cm}{3.5cm}{*}
%\setulmarginsandblock{2cm}{2cm}{*}
%\checkandfixthelayout

\subimport{sections/}{section4.tex}

%--------------------------------------------------------
%--------------------------------------------------------
%--------------------------------------------------------


\subimport{sections/}{section5.tex}

%--------------------------------------------------------
%--------------------------------------------------------
%--------------------------------------------------------

%\printbibheading
%\printbibliography[section=1,heading=subbibliography]
%\printbibliography[heading=subbibliography]
%\printbibliography[heading=bibnumbered]
%\printbibliography
\printbibliography[segment=\therefsegment,heading=subbibliography]

%%% Local Variables:
%%% mode: latex
%%% TeX-master: "masterproef"
%%% End:

